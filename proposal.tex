\documentclass[draft]{ubcstatproposal} % remove draft option when done
 % \usepackage{shortex} add this if you like, https://github.com/trevorcampbell/shortex.git

% define theorem, corollary, lemma, and definition environments
\theoremstyle{plain}
\makeatletter
\@ifundefined{theorem*}{\newtheorem*{theorem*}{Theorem}}{}
\@ifundefined{theorem}{\newtheorem{theorem}{Theorem}}{}
\@ifundefined{corollary}{\newtheorem{corollary}[theorem]{Corollary}}{}
\@ifundefined{lemma}{\newtheorem{lemma}[theorem]{Lemma}}{}
\@ifundefined{definition}{\newtheorem{definition}[theorem]{Definition}}{}
\makeatother


\title{Your Document Title}
\author{Your Name}
\date{Date of Your Proposal/Presentation}

% Your advisor. If you have a coadvisor, just use \advisor twice.
\advisor{John Snow}

% Your committee or external advisors. For external advisors or external
% committee members, be sure to include their affiliation.
\committee{Florence Nightingale, King's College, London}
\committee{Henry Scheffé}
\committee{John Tukey}

% Choose one:
\documenttype{Thesis Proposal} % for thesis proposals

% If your title is very long, it may not fit in the page headers. You can define
% a short version for those:
% \shorttitle{Your Title}

% The default font is Latin Modern, aka Computer Modern. You may choose a
% different font by uncommenting one of the following lines. All of these
% options include matching mathematics fonts.
% \usepackage{libertinus} % Libertinus (requires LaTeX from 2019 or newer)
% \usepackage{stix2} % STIX Two (also requires recent LaTeX)
% \usepackage{newtxtext,newtxmath} % Times New Roman style font
% \usepackage{newpxtext,newpxmath} % New PX, based on Palatino

% Put your custom commands here:
% For example,
% \DeclareMathOperator{\E}{\mathbb{E}}
% \DeclareMathOperator{\logit}{logit}

\begin{document}
\maketitle


\begin{abstract}
  Provide an abstract here. A good abstract is concise, self-contained, and
  readable by any member of the department.
\end{abstract}

\section{Introduction}

This is a template for thesis proposals in the Department of Statistics at The
University of British Columbia. Students should also read the Comprehensive
Examination Guidelines for details on the expected contents of reports and
proposals, as well as the requirements for the presentations and timeline.

This template requires a basic knowledge of \LaTeX\ and should cover the basic
requirements in terms of required packages and functionality. There are a few
useful resources if you need \LaTeX\ help:
\begin{itemize}
\item The \href{https://en.wikibooks.org/wiki/LaTeX}{LaTeX Wikibook}, a full
  guide to \LaTeX
\item The
  \href{http://mirrors.ctan.org/macros/latex/required/amsmath/amsldoc.pdf}{amsmath
    manual} describes the mathematical features provided by the amsmath package:
  aligned equations, multiline equations, matrices, arrows, special symbols and
  characters, and so on.
\end{itemize}

We recommend using \BibTeX\ for your references. This template is set up to use
\href{http://mirrors.ctan.org/macros/latex/contrib/natbib/natbib.pdf}{the natbib
  package} for bibliography management, using the ``apalike'' reference style.
You can adjust this to any referencing format you prefer.

The \texttt{proposal-references.bib} file contains some example \BibTeX\ entries.
To cite, use the \verb|\citet| command to use a reference in a sentence, and use
\verb|\citep| for a parenthetical citation. For example, \citet{Wasserman:2004}
is a common statistics reference, and I've cited it here with \verb|\citet|.
Alternately, we often put references in parentheses using \verb|\citep|
\citep{Underhill:1999}.

Here's some dummy text to push the figure down. \lipsum[2]

Figures, such as Fig.~\ref{fig:label}, go into the \texttt{figures/} folder. This
template uses the \verb|cleverref| package so that you can do things like
\verb|\cref{fig:label}| to produce \cref{fig:label}. For more details, see
\href{https://ctan.mirror.globo.tech/macros/latex/contrib/cleveref/cleveref.pdf}%
{the \texttt{cleveref} manual}. You may also wish to consider using 
\href{https://github.com/trevorcampbell/shortex.git}{Shor\TeX}.

\begin{figure}
  \centering
  \includegraphics[width=0.3\textwidth]{fig02a-circle}
  \caption{Sample caption.}
  \label{fig:label}
\end{figure}

Given a general quadratic of the form $ax^2 + bx + c$, the roots may be found by
the quadratic formula,
\[
  x = \frac{-b \pm \sqrt{b^2 - 4ac}}{2a}.
\]

\subsection{Jumbo-sized figures and tables}

If you have a figure or table that does not fit in a normal-sized page, such as
a very wide table with many columns, you can create a landscape page. The
landscape page is rotated (11 inches wide and 8.5 inches tall, instead of the
reverse), so wide things fit in it. This is ugly however and best avoided.

The \verb|landscapepage| environment does this. Simply place your figure or
table environment inside a \verb|landscapepage| environment and it will appear
on its own page, in landscape. See Table~\ref{jumbo-table} for an example. The
table will be correctly rotated in PDF readers, so it's easy to read on screen,
but will print out correctly on ordinary paper.

By default, the landscape page will have 1-inch margins. You can adjust this
with an optional argument to the environment, e.g.\ here the margins will be 1
cm:
\begin{verbatim}
\begin{landscapepage}[1cm]
  \begin{table}
    ...
  \end{table}
\end{landscapepage}
\end{verbatim}
You should leave at least \emph{some} margin, since most printers cannot print
all the way to the edge of the page.

\begin{landscapepage}
  \begin{table}
    \centering
    \begin{tabular}{l r}\toprule
      Foo & Bar \\\midrule
      A & 1.2\\
      B & 2.3\\\bottomrule
    \end{tabular}
    \caption[Demonstration extra-wide table]{A table that's too big to fit on a
      standard-sized page.}
    \label{jumbo-table}
  \end{table}
\end{landscapepage}

\section{Prior literature}

\section{Proposed work}

These section headings are meant only as a rough outline for thesis proposals;
you can determine the structure of your proposal however you and your advisors
feel is appropriate, provided it meets the requirements in the Ph.D.\ handbook.


\section{Timeline}

\bibliographystyle{apalike} % bibliography style - recommend using apalike-doi
% as it hyperlinks DOIs
\bibliography{proposal-references}

\end{document}
